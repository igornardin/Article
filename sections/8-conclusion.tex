This study aimed to raise and discuss the main issues regarding microservice metrics and scalability. To achieve this, we selected two general research question and four specific research question that directs our article. We used these questions to create the research method. We created the search string based on research questions that we submitted to electronic databases. We classify and filter the results obtained from the database. We used the result from this filtering to create a taxonomy to describe the microservice evaluation. This microservice taxonomy can describe the application class, the evaluation metrics, architecture, evaluation algorithm, communication protocol, communication pattern, and the virtualization. 

After determining the taxonomy, we answered each research question and presenting the tendencies of the questions individually. These tendencies can help future developments with microservice concerning performance and scalability. Following analyzing the research question, we correlated the research questions showing other tendencies, like the metric most utilized in the application types. Finally, we presented some opportunities that we did not find in the reviewed articles.

In future studies, we will focus on the opportunities to implements a new solution to microservice performance and scalability. We presented some opportunities in the previous section that we did not found in reviewed articles, like improving ERP and image processing application type, vertical elasticity utilization, mobile agents architecture, proactive algorithms to batch applications, and energy motivation.