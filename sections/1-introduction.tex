At the beginning of software development, the developers implemented their software with the monolithic approach. In this approach, a single executable compacts all functionalities. Therefore, even if the user only needs one function, he will receive all the functions package of that software. When this application is on the internet, can be difficult to improve performance. For example, one function has an increase in its requisitions, so, to provide performance and scalability, the system needs to add more software instances. In the monolithic approach, the new instance replicates all the application, when only part of the application was required. An evolution of this approach is microservice architecture.

Big companies, like Netflix, Amazon, and The Guardian use the microservice architectures in their solutions ~\cite{Francesco2017ResearchAdoption}. In the microservice architecture, the developer decomposes the software into a set of small services, each running in its process and using a lightweight mechanism to communication (like REST)~\cite{Fowler2014Microservices}. Each service has a clear function and performs just this function, and because of this definition that microservice has this nomenclature. Decomposing the software allows that each microservice use the better framework to implements its purpose. For example, the programmer can use the Java language to implements a user interface microservice and R language to process complex calculations. Besides that, this decomposition enables the scaling of just one software microservice. So, if one microservice has an increase in its requisitions, just this microservice will replicate to a new instance.

A microservice architecture facilitates the adjustments in the functions, but this approach will not make these improvements by itself. The software manager needs to analyze and decide when adjusting the application environment. Decide the best moment to improve the application is not an easy decision. For example, a web application can have a peak moment of usage. Make a decision based only on this peak can instantiate a resource inappropriately. Besides that, if the manager takes too long to instantiate a new resource when an undesirable state reaches may affect the application performance. Therefore, it is essential to describe some architectures, metrics, patterns, and protocols for microservice performance and scalability bias.

Some articles aim to demonstrate some tendencies and opportunities to microservice. In~\cite{Almeida2017SurveyEnvironment} the authors presented a survey about microservice architecture focusing on security, privacy and standardization aspects. This article analyzes just microservices in cloud computing environments. Already the article~\cite{Francesco2017ResearchAdoption} consists of a systematic mapping study in three perspectives: publication trends, the research focus, and potential for industrial adoption. Like~\cite{Almeida2017SurveyEnvironment}, in~\cite{Yu2018AApplications} the primary motivation is security, but in this article, the focus is on the communication between microservices. This article presents some concerns about microservice communication and in the end, demonstrate an ideal solution for security issues. The article~\cite{Alshuqayran2016AArchitecture} presents a systematic mapping study to find architectural challenges, architectural diagrams/views, and quality attributes. Finally, the article~\cite{Cerny2018ContextualArchitecture} presents a systematic mapping study to identify interest and challenges in microservices. 

None of the above articles focus on microservice performance and scalability or how to improve this. Therefore, this survey aims to present update review about tendencies and opportunities for microservice performance and scalability. So, we define research questions to find motivations, metrics, algorithms, application classes, architectures, and communication patterns and protocols about microservice evaluation. With the results, we proposed a taxonomy to direct future implementations in microservices. Therefore, we will show some tendencies and opportunities found in the reviewed articles.

The rest of this paper is structured as follows: Section~\ref{background} explains some concepts about microservice and cloud computing (the major microservice case). Section~\ref{methods} explains the research method that we use in this study. Section~\ref{results} shows the results of our research. In section~\ref{Discussion}, we present a discussion about the tendencies of microservice performance and scalability. Section~\ref{opportunities} demonstrates some opportunities for future works in microservice that we did not find in our research. Finally, in section~\ref{conclusion} we conclude and present future studies.